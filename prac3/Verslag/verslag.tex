\documentclass{article}

\usepackage[dutch]{babel}
\usepackage{epsfig}
\usepackage{verbatim}
\usepackage{moreverb}
\usepackage{float}
\usepackage{graphicx}
\usepackage{framed}
\usepackage{chngcntr}
\usepackage{enumitem}
\usepackage{verbdef}
\usepackage[font=bf]{caption}

\author{Peter van Dijk s1102109 \& Elizabeth Schermerhorn s1223380}
\date{\today}
\title{Practicum 2 Synchronisatie bij vuurvliegjes}
\begin{document}
\maketitle
\newpage
\tableofcontents
\clearpage
\section{Inleiding}
In de natuur zijn er verschillende vormen van synchronisatie. Aangezien synchronisatie een belangrijk punt is in de werking tussen verschillende nodes in een netwerk is het van belang om een goed algoritme hiervoor te hebben. Een bekend voorbeeld uit de natuur zijn vuurvliegjes die met elkaar synchroniseren, wat zich uit in samen knipperen. Het algoritme dat vuurvliegjes gebruiken om synchroon te knipperen wordt ge\"{i}mplementeerd in een multi-hop netwerk. Door het algoritme te gebruiken moet het mogelijk zijn om nodes tegelijk te laten knipperen net zoals een groep vuurvliegjes. 
\section{Probleemstelling}
Het doel van dit onderzoek is het vinden van een algoritme om Arduino's te synchroniseren met behulp van radio's. Een aantal eisen waar dit algoritme aan moet voldoen zijn:
 \begin{itemize}
 \item Wanneer twee gesynchroniseerde netwerken worden samengevoegd dienen ze te synchroniseren.
 \item Wanneer een node uitvalt dient de synchronisatie nog te werken.
 \item Wanneer nodes toegevoegd worden aan het netwerk gaan deze met dezelfde frequentie knipperen als de andere nodes in het netwerk.
 \item Wanneer nodes uit synchronisatie raken moeten ze opnieuw synchroniseren. 
 \end{itemize}
Nu is vastgesteld aan welke eisen het algoritme moet voldoen kan er een hypothese opgesteld worden. Om ervoor te zorgen dat nieuwe nodes met dezelfde frequentie gaan knipperen als nodes die al langer in het netwerk verkeren zal er om de zoveel tijd opnieuw met alle nodes gesynchroniseerd moeten worden. Door dit in de implementatie toe te voegen zou dit ook moeten werken. Dit geldt ook voor de drie resterende bovengenoemde punten. 

\section{Gerelateerd werk}
Er zijn verschillende onderzoeken gedaan naar het onderwerp synchronisatie in een multi-hop netwerk. 
Een paar papers welke dit onderwerp aansnijden zijn: 
\begin{itemize}
	\item Clock synchronization for wireless sensor networks: a survey
	\item Wireless sensor network survey
	\item Academic Press Library in Signal Processing, Chapter 2 – Synchronization
\end{itemize}
De bovenstaande drie papers gaan over synchronisatie in draadloze netwerken. De eerste paper - Clock synchronization for wireless sensor networks: a survey - evalueert bestaande kloksynchronisatie-algoritmen gebaseerd op factoren zoals precisie, complexiteit, nauwkeurigheid en kosten \cite{survey}. De tweede paper, Wireless sensor network survey, geeft een overzicht van bestaande draadloze netwerk implementaties op verschillende niveaus \cite{survey2} en als laatste geeft het boek Academic Press Library in Signal Processing een kort overzicht van tijd synchronisatie in een netwerk \cite{academic}. \\
\\
In Sectie 3.3 van de paper "Clock synchronization for wireless sensor networks: a survey" wordt gesproken over de verschillende mogelijkheden bij de implementatie en hoe deze in hun werk gaan. 
Wanneer er een draadloos sensor netwerk ontwikkeld wordt zijn er een aantal factoren waar rekening mee gehouden moet worden. Hieronder staat een lijst weergegeven met de verschillende design-keuzes die gemaakt moeten worden.
\begin{itemize}
	\item Master-slave protocol versus peer-to-peer synchronisatie
	\item Interne synchronisatie versus externe synchronisatie
	\item Single-hop netwerk versus multihop netwerk
	\item probabilistische versus deterministische synchronisatie
\end{itemize}
In Sectie 3.3 van de paper "Clock synchronization for wireless sensor networks: a survey" staan nog meer keuzemogelijkheden, deze zijn niet van belang voor dit experiment en zijn voor het gemak weggelaten. 
In Appendix \ref{Terminologie} staat kort beschreven wat elke implementatiemogelijkheid inhoudt.\\
\\
Afhankelijk van de eisen die aan het netwerk gesteld worden moet er voor bepaalde onderdelen gekozen worden. Eisen aan het netwerk zouden kunnen zijn: Er is geen server beschikbaar, het moet zo min mogelijk energie verbruiken zodat de nodes lang meegaan of de nodes moeten over grotere afstand kunnen communiceren. Dit zijn slechts een aantal voorbeelden. \\
\\
In de tweede genoemde paper, Wireless sensor network survey, worden in sectie 6.2 meerdere synchronisatie-algoritmen uitgelegd. De volgende algoritmen worden beschreven:
\begin{itemize}
	\item Onzekerheid gedreven benadering
	\item Lucarelli's algoritme
	\item Vuurvlieg algoritme
	\item Tijd synchronisatie protocol voor sensor netwerken
	\item klok-opname gezamenlijke netwerk synchronisatie
	\item Tijd synchronisatie
	\item globale synchronisatie
	synchronisatie 
\end{itemize}
Het ge\"{i}mplementeerde algoritme dat in deze paper besproken wordt is gebaseerd op het vuurvliegalgoritme. Hoewel het ge\"{i}mplementeerde algoritme erop gebaseerd is, is het op sommige punten aangepast. 
Het vuurvliegalgoritme werkt als volgt: 
Elke node in het netwerk werkt als een oscillator met een vastgestelde tijd $T$. Elke node heeft een interne tijd $t$ die opgehoogd wordt totdat deze gelijk is aan $T$. Op tijd $T$ zal de node een signaal sturen en zijn interne tijd $t$ weer op $0$ zetten. Buren van deze node die dit signaal ontvangen zullen de tijd tussen $t$ en $T$ verkleinen. Dit wordt bepaald door middel van een functie en een kleine constante. Na verloop van tijd zullen de nodes gelijk hun signaal versturen dus gesynchroniseerd zijn. De bedoeling is dat alle nodes tegelijk een signaal gaan versturen na verloop van tijd.  
\begin{figure}[h]
\centering\includegraphics[scale=0.5]{sync.png}
\caption{Tijdlijnen van twee synchroniserende nodes}
\label{Onze_implementatie}
\end{figure}
Het grote verschil tussen bovenstaande implementatie en de experimentele implementatie van deze paper is dat de $t$ niet naar $0$ wordt gezet maar het verschil tussen het luisteren naar berichten en het uitlezen van de berichten wordt bijgehouden. Op basis van deze tijd wordt er voor een bepaalde tijd geslapen waardoor het tijdsverschil afneemt. Door dit fenomeen synchroniseren na verloop van tijd de nodes in het netwerk. 
\ref{fig: Onze_implementatie}

\section{Ontwerp synchronisatie-algoritme}
Het synchronisatie-algoritme is gebaseerd op een master-slave structuur. In Figuur \ref{fig: Schematisch} staat een schematische weergave van het netwerk zoals deze is getest. Er zijn ID's meegegeven aan de nodes zodat er te zien is dat er \'{e}\'{e}n master is met \'{e}\'{e}n of meerdere slaves. Het algoritme dat deze nodes gebruiken voor de synchronisatie staat in pseudocode in Figuur \ref{fig: pseudo}. Verder is de synchronisatie tussen twee nodes visueel weergegeven in Figuur \ref{fig: Onze_implementatie}.
\begin{figure}[h]
\centering\includegraphics[scale=0.5]{testopstelling}
\caption{Schematische weergave van de testopstelling}
\label{fig: Schematisch}
\end{figure}
\begin{figure}
\centering
\verbdef\demo{demonstration text}
\framebox{
 \begin{minipage}{14.5cm}
\centering
\verbatiminput{./code/pseudo.txt}
 \end{minipage}
}
\caption{Pseudoalgoritme}
\label{fig: pseudo}
\end{figure}
Iedere node in het netwerk krijgt een willekeurig ID tussen de $0$ en de $1000$. De master van het netwerk wordt gekozen aan de hand van het hoogste ID. Er is gekozen voor deze structuur aangezien er gesynchroniseerd moet worden op een gekozen node. Het doel is om de nodes tegelijk te laten knipperen. Om dit te doen moeten de interne klokken van de nodes gelijk lopen. In onze implementatie detecteert een node het moment dat de master wakker is en zal zijn interne klok corrigeren naar dit moment. Binnen deze master-slave structuur is de implementatie verdeeld in twee alternerende delen.\\
\\
In het tweede deel wordt een aantal maal achtereen alternerend de Arduino en Radio in slaapstand gezet en vervolgens een bericht verstuurd. De lengte van deze slaapstand, of `nacht' staat vast voor het hele netwerk.\\
\\
In het eerste deel wordt er gesynchroniseerd, eerst wordt er voor 
\'e\'en nacht geluisterd of er berichten zijn ontvangen. Als er berichten zijn ontvangen worden deze verwerkt. Dan wordt er voor een tijd, die afhankelijk is van de ontvangen data, geslapen en ten slotte wordt er een bericht gestuurd. \\
\\
Eerst zullen de gebruikte instellingen samen met het verstuurde signaal besproken worden. Daarna zullen de twee delen waaruit de implementatie bestaat toegelicht worden. 

\subsection{Gebruikte instellingen}
De RF24-radio wordt gebruikt met de volgende instellingen:
\begin{itemize}
  \item geen hertransmissies
  \item pakketgrootte: grootte van bericht
  \item geen automatische bevestiging bij ontvangen bericht
\end{itemize}
Verder wordt er gebruik gemaakt van de standaardinstellingen. Het kanaal waarop gezonden wordt is hetzelfde kanaal als waarop geluisterd wordt, waardoor een node dus \'of aan het luisteren is \'of aan het zenden, aangezien de radio niet beiden tegelijk kan op het zelfde kanaal.\\
\\
Het uitzetten van hertransmissies zorgt ervoor dat er geen berichten worden gestuurd op momenten die niet het wakker worden van de node aanduiden. Als deze hertransmissies wel gestuurd zouden worden, dan zou er gesynchroniseerd kunnen worden op hertransmissies, wat niet op het juiste tijdstip zou zijn. Het uitzetten van de automatische bevestiging zorgt er voor dat er geen onnodige berichten worden gestuurd, immers zal er niets gedaan worden met de bevestiging, aangezien berichten niet opnieuw gestuurd worden.

\subsection{Berichten}
Berichten die verstuurd worden bevatten de volgende data:
\begin{enumerate}
\item ID van de zender
\item hoogst door zender ontvangen ID
\item ID van bericht
\end{enumerate}
De hoogste ID is nodig om te beslissen of een bepaalde node de master node is en of daarnaar geluisterd moet worden. 
De ID van de zender toegevoegd om te kunnen achterhalen welke node het bericht heeft gestuurd. \\
\\
Er is voor gekozen berichten een ID te geven, waarbij de combinatie tussen hoogst ontvangen ID en ID van het bericht uniek is binnen het netwerk. Hierdoor kunnen veroudere berichten gedetecteerd worden, zodat deze niet gebruikt worden om op te synchroniseren. Dit voorkomt dat nodes een ID van een uitgevallen node blijven versturen als hoogst gevonden ID.

\subsection{Het eerste deel, synchroniseren}
Het eerste deel bestaat uit alternerend slapen en berichten versturen. In het eerste deel van het programma, zoals aangegeven in Figuur \ref{fig: pseudo}, wordt voor een nacht gekeken welke andere nodes er in het netwerk aanwezig zijn. \\
\\
Op het moment dat er begonnen wordt met luisteren wordt er een tijdstip $t_0$ geklokt. Op het moment dat er een bericht van de node met het hoogst gedetecteerde ID wordt ontvangen wordt er een tweede tijdstip $t_1$ geklokt. $\Delta t$, oftewel $t_1 - t_0$, is het tijdsverschil tussen de interne klokken van de Arduino's. Daarom wordt de tweede nacht van de synchronisatie voor $\Delta t$ geslapen. Na deze nacht zijn de nodes gesynchroniseerd en zal de node een bericht sturen met het grootst gevonden ID. Na de eerste nacht wordt geen bericht gestuurd, omdat de node dan nog niet gesynchroniseerd is.

\subsection{Het tweede deel, slapen en versturen}
Als de nodes gesynchroniseerd zijn zullen ze alternerend een nacht slapen en een bericht sturen. Deze berichten worden gestuurd zodat ongesynchroniseerde nodes kunnen synchroniseren op de ontvangst van deze berichten.

\section{Testopstelling en resultaten van de test}
In deze paragraaf wordt het ge\"{i}mplementeerde algoritme onderzocht op hoe snel de synchronisatie plaatsvindt bij verschillende afstanden. De hypothese is dat naarmate de afstand groter wordt de tijd om te synchroniseren toeneemt. Er is meer tijd nodig om de berichten te versturen dus zal het langer duren voor het synchronisatie plaatsvindt. De schematische opstelling hiervan is te zien in Tabel \ref{tab: testresultaten} en de re\"{e}le opstelling zoals deze was is te zien in Figuur \ref{fig: Nodes_tijdens_testen}. 
\begin{figure}[h]
\centering\includegraphics[scale=0.09, angle=90]{Nodes_tijdens_testen}
\caption{Testopstelling}
\label{fig: Nodes_tijdens_testen}
\end{figure}
\begin{table}[h]
	\centering\caption{Testresultaten}
	\label{tab: testresultaten}
    \begin{tabular}{| l | l | l | l |  l |}\hline
    \textbf{afstand } & \textbf{10 cm } & \textbf{25 cm} & \textbf{50 cm} & \textbf{100 cm} \\ \hline\hline
   1. & 4,33 seconden & 11,50 seconden& 7,40 seconden& 6,72 seconden\\ \hline
    2. & 4,72 seconden & 5,58 seconden& 17,46 seconden& 6,18 seconden\\ \hline
    3. & 4,07 seconden & 9,0 seconden& 8,84 seconden& 6,40 seconden\\ \hline
    4. & 3,82 seconden & 9,85 seconden& 12,54 seconden& 5,71 seconden\\ \hline
    5. & 7,92 seconden & 4,20 seconden& 8,87 seconden& 4,93 seconden\\ \hline \hline
   gemiddelde & 4.972 seconden & 8.026 seconden& 11.22 seconden& 5.988 seconden\\\hline
    \end{tabular}
\end{table}
In Tabel \ref{tab: testresultaten} is te zien hoe de afstand van de nodes zich verhoudt tot de tijd die het neemt om te synchroniseren. Wat opmerkelijk is aan deze tabel is dat met een afstand van 50 centimeter de synchronisatietijd soms meer dan twee keer zo groot is als wanneer er op een afstand van 100 centimeter wordt gesynchroniseerd. Twee mogelijke verklaringen voor dit fenomeen zijn:
\begin{itemize}
	\item Door interferentie met Wi-Fi is er op bepaalde momenten soms meer of minder bereik van de radio's.
	\item door de gebruikte golflengtes van de radio kan het zijn dat er dode punten ontstaan, zoals dit ook het geval is bij satellieten. Er zijn bepaalde afstanden waar dit voor gebeurt en dit zou ook kunnen bij bijvoorbeeld 50 centimeter afstand. 
\end{itemize}
\section{Conclusie en aanbevelingen}
De probleemstelling die gesteld werd is: \textit{"het vinden van een algoritme om Arduino's te synchroniseren met behulp van radio's"}. Na het testen van de implementatie kan er geconcludeerd worden dat deze de Arduino's laat synchroniseren en het voorgestelde algoritme dus werkt. In Tabel \ref{tab: Gestelde eisen} staan de eisen waar de implementatie aan voldoet. In het hoofdstuk \textit{Ontwerp synchronisatie-algoritme} en in \textit{Testopstellingen en resultaten} staan de implementatie en de testresultaten. 
\begin{table}[h]
	\centering\caption{Gestelde eisen aan het netwerk}
	\label{tab: Gestelde eisen}
	\begin{tabular}{|l|p{10cm}|}\hline
	\textbf{nummer} & \textbf{eis} \\ \hline
	1. & Wanneer twee gesynchroniseerde netwerken worden samengevoegd 
	dienen ze te synchroniseren. \\ \hline
	2. & Wanneer een node uitvalt dient de synchronisatie 
	nog te werken. \\ \hline
	3. & Wanneer nodes toegevoegd worden aan het netwerk gaan deze met dezelfde
	 frequentie knipperen als de andere nodes in het netwerk.\\ \hline
	4. & Wanneer nodes uit sync raken moeten ze opnieuw synchroniseren. \\ \hline
	\end{tabular}
\end{table}
\newline
Hoewel de code werkt en de tests zijn gelukt staat de implementatie open voor verbetering. Tijdens dit onderzoek was er slechts de beschikbaarheid over drie nodes. In de subsectie aanbevelingen worden verder testmethoden en uitbreidingen voorgesteld. 
\subsection{Aanbevelingen}
Tijdens dit onderzoek was er slechts de beschikbaarheid over drie nodes. In een vervolgonderzoek kan er meer uitgebreid getest worden of het ook werkt met veel meer nodes. Vragen die hierbij gesteld kunnen worden zijn:
\begin{itemize}
	\item Is er een maximum aan het aantal nodes in het netwerk waarop er nog gesynchroniseerd kan worden? Aangezien er gebruik wordt gemaakt van een willekeurig ID tussen 0 en 1000 is het van belang om te weten wanneer er vaak dubbele ID voorkomen. 
	\item Verandert de synchronisatietijd die nodig is om te synchroniseren naarmate er meer nodes in het netwerk aanwezig zijn?
	\item De mogelijkheden onderzoeken om in plaats van een gedecentraliseerd netwerk een gecentraliseerd netwerk te synchroniseren, gebruik makende van de code met enkele aanpassingen. 
\end{itemize}
\newpage
%\section{References}
%\nocite{*} % Even non-cited BibTeX-Entries will be shown.
\bibliographystyle{abbrv}
\bibliography{literature}



\clearpage
\appendix
\counterwithin{figure}{section}
\section{Code Sender}
\verbatiminput{./code/node.ino}
\newpage

\section{Terminologie}

\label{Terminologie}
\begin{table}[H]
	\centering
	\caption{Toelichting op de implementatie mogelijkheden}
	\label{tab: implmentatie_mogelijkheden}
    \begin{tabular}{ | l | p{5cm} |}\hline
    
    \textbf{Onderdeel} & \textbf{Uitleg} \\ \hline\hline
    Master-slave protocol & Er is een master node in het netwerk waar alle andere nodes naar luisteren, deze bepaald de synchronisatie \\ \hline
    peer-to-peer protocol &  Elke node in het netwerk heeft contact met elke andere node. Dit elimineert het risico dat wanneer een master node wegvalt het netwerk niet meer synchroniseert.\\ \hline
    Interne synchronisatie & De re\"{e}le tijd is niet beschikbaar. Het doel is het verschil in tijd tussen de lokale klok en het uitlezen van de sensor zo klein mogelijk te maken.\\
    \hline
    externe synchronisatie & Hier is de tijd wel beschikbaar zoals UTC. Er is een atomische klok die de ware tijd heeft.\\
    \hline
    single hop netwerk & Elke node in het netwerk heeft contact met elke andere node in het netwerk\\
    \hline
    multi hop netwerk & Niet elke node in het netwerk heeft contact met een andere node in het netwerk. \\
    \hline
    probabilistische synchronisatie & Geeft een probabilistische garantie van de maximum offset van de klok. \\
    \hline
    deterministische synchronisatie & Deze algoritme garanderen een bepaalde maximum offset van de klok. \\
    \hline
    \end{tabular}
\end{table}

\end{document}
